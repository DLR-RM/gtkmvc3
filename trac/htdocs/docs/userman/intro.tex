\section{Introduction}

This document contains information about the functionalities and the
architecture of a Model-View-Controller and Observer Infrastructure
(\mvco from here on) for \pygtk version 2. The aim is to supply the
essential information in order to make developers able to effectively
use, modify and extend the infrastructure, for easily creating
new applications based on it.

\bigskip

Section \ref{MOT} tries to motivate the usage of \pygtkmvc, by
discussing the context which the framework is thought to operate, and
its main goals.

Section \ref{ARCH} briefly gives an overview of the general
architecture for a \gui application based on Python and the GTK
toolkit, showing all major parts, and how these depend on each other.
In this general picture, the proposed \mvco framework is also
collocated, to provide an initial idea of how an hypothetical
application based on it should be organized.

Section \ref{MVCO} describes the basement of a \gui application, the
\mvco Infrastructure. 

Sections \ref{DI} and \ref{SAP} supply some further details
about implementation, via many examples. The examples aim to make more
concrete the ideas described in all previous sections.

Section \ref{ADAPT} presents a new important feature provided since
version 1.2.0, \emph{Adapters} that make easier exploit the
framework when a default behaviour is expected. 

Finally, section \ref{PROGEN} briefly discusses how a project based on
\mvco can be easily created from scratch by using a little application
that is provided with version 1.2 and later.
