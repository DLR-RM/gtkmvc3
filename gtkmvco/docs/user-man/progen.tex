\section{progen: A Project Generator}

Since version 1.2.0 a little application called \kw{gtkmvc-progen}
is provided. Goal of \kw{gtkmvc-progen} is to generate the skeleton
of a project that can be used when starting up a new application
based on gtkmvc.

\kw{gtkmvc-progen} creates a directory containing the skeleton of a
new project, called the top-level directory.

The newly created project is constituted by:
\begin{enumerate}
\item A source directory containing all project source code
\item An empty application-level Model
\item A simple application-level Controller
\item The application main View containing the application main window. 
\item A resources directory, containing for example the project
  glade files
\item A launching script, localized in the top-level directory
\end{enumerate}


\kw{gtkmvc-progen} can be executed as a batch program to be
controlled from the command line, or as a simple GUI
application. \kw{gtkmvc-progen} is of course based on \pygtkmvc. All
the work is performed by the model, and a view and a controller are
loaded when a GUI is required. Depending on the hosting platform,
\kw{gtkmvc-progen} is launched by default either in batch or in GUI
mode. Unix users are more familiar with command-line programs, and
will find \kw{gtkmvc-progen} to be executed in that modality by
default. Windows users will find the GUI presented by default
instead.

The way \kw{gtkmvc-progen} can be customized is by setting some
properties inside the model, either by using the command line, or by
using the GUI that does not export the full set of properties,
though.

Here is the list of properties with a short description:

\begin{center}
\begin{tabular}{|l|l|l|l|}
\hline
Property name& Property type & Description & Default value\\[0.5ex]
\hline
name & string & name of the project & \textbf{REQUIRED} \\
author & string & Developer's name & \textbf{REQUIRED} \\
email & string & Developer's email address & \\
copyright & string & Copyright string & A sensible string\\
destdir & string & name of destination directory & "." \\
complex & bool & Generates hierarchical MVC support & True \\
dist\_gtkmvc & bool & If True, gtkmvc is embedded & True \\
glade & bool & if glade files are going to be used or not & True \\
\hline
glade\_fn & string & filename of generated glade file & application.glade \\
src\_header & string,None & Template for source header files. & None \\
other\_comment & string & Additional comment pushed after headers &  \\ 
src\_name & string & Name of the source directory & "src" \\
res\_name & string & Name of the resources directory & "resources" \\
top\_widget & string & Name of the View's top-level widget & "window\_appl" \\
\hline
\end{tabular} 
\end{center}

Bottom part of the table contains less important properties. Python
module \file{gtkmv.progen.templates} contains default templates that
are used for headers, license, etc. 

Boolean option \kw{gui} can be used to select batch or gui
mode. Option \kw{help} can be used to print out an helping message.

\bigskip

\kw{gtkmvc-progen} can be executed either locally from the script
directory, or can be executed as any other program if \pygtkmvc has
been officially installed on the hosting system.

From the local script directory:
\begin{verbatim}
$> python gtkmvc-progen param=val ...
\end{verbatim}

If \pygtkmvc was installed:
\begin{verbatim}
$> gtkmvc-progen param=val ...
\end{verbatim}

Boolean properties can be specified in the form ``param'' or in the
form ``param=[yes|no]''. Specifying boolean ``param'' or
``param=yes'' is semantically equivalent.

For example:
\begin{verbatim}
$> gtkmvc-progen name=hello author="Roberto Cavada" gui glade=no
\end{verbatim}

The result is the creation of the top-level directory whose name is
the project name. Inside a top-level script can be used to launch
the application. 
